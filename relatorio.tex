%%%%%%%%%%%%%%%%%%%%%%%%%%%%%%%%%%%%%%%%%
% Simple Sectioned Essay Template
% LaTeX Template
%
% This template has been downloaded from:
% http://www.latextemplates.com
%
% Note:
% The \lipsum[#] commands throughout this template generate dummy text
% to fill the template out. These commands should all be removed when 
% writing essay content.
%
%%%%%%%%%%%%%%%%%%%%%%%%%%%%%%%%%%%%%%%%%

%----------------------------------------------------------------------------------------
%	PACKAGES AND OTHER DOCUMENT CONFIGURATIONS
%----------------------------------------------------------------------------------------

\documentclass[11pt]{report} 
\usepackage{geometry} 
\geometry{a4paper} 
\usepackage{graphicx} 
\usepackage{float} 
\usepackage{wrapfig} 
\linespread{1.2} % Line spacing
%\setlength\parindent{0pt} % Uncomment to remove all indentation from paragraphs
\graphicspath{{Pictures/}} % Specifies the directory where pictures are stored
\usepackage[portuguese]{babel} % Uncomment to turn to portuguese
\usepackage{listings}
\usepackage{ragged2e}
\usepackage[utf8]{inputenc}
\usepackage{color}
\usepackage{fancyhdr} % for headers and footers


%\righthyphenmin=62
%\lefthyphenmin=62
 
\definecolor{codegreen}{rgb}{0,0.6,0}
\definecolor{codegray}{rgb}{0.5,0.5,0.5}
\definecolor{codepurple}{rgb}{0.58,0,0.82}
\definecolor{backcolour}{rgb}{0.95,0.95,0.92}
 
\lstdefinestyle{mystyle}{
    backgroundcolor=\color{backcolour},   
    commentstyle=\color{codegreen},
    keywordstyle=\color{magenta},
    numberstyle=\tiny\color{codegray},
    stringstyle=\color{codepurple},
    basicstyle=\footnotesize,
    breakatwhitespace=true,         
    breaklines=true,                 
    captionpos=b,                    
    keepspaces=true,                 
    numbers=left,                    
    numbersep=5pt,                  
    showspaces=false,                
    showstringspaces=false,
    showtabs=false,                  
    tabsize=8
}

\lstset{style = mystyle}

\begin{document}

 
% headers
\fancyhead[L]{Relatório do Projeto}
\fancyhead[C]{Mastermind}
\fancyhead[R]{LEIC-A}
% footers
\fancyfoot[L]{Introdução à Arquitectura de Computadores}
\fancyfoot[C]{}
\fancyfoot[R]{Instituto Superior Técnico}

% need to specify the pagestyle as fancy
\pagestyle{fancy}
 
\newcommand{\HRule}{\rule{\linewidth}{0.5mm}} % Defines a new command for the horizontal lines, change thickness here


% Information

\begin{center} \large
\includegraphics[width = 3cm] {/home/bdinis/Documents/IST/ist.jpg}\\
{Baltasar \textsc{Dinis}, 89416} \\
{Vasco \textsc{Rodrigues}, 89557} \\
\end{center}

%%%%


\section*{Descrição Geral do Projeto}
O programa implementado no projeto permite que o utilizador jogue o
\textit{Mastermind} na placa ou no simulador do P3. Neste jogo, ao longo
de um determinado número de rondas, o jogador introduz uma jogada, em
que tenta adivinhar um código secreto, gerado pelo programa. A cada
jogada o programa indica o número de dígitos certos na posição certa e o
número de dígitos certos na posição errada. A pontuação corresponde ao número
mínimo de jogadas necessárias para descobrir o número secreto, e é inicializada
com o valor seguinte ao máximo de rondas que se pode jogar, garantindo que na 
primeira ronda é atualizada.

%A interação é feita através dos botões de interrupção, que permitem introduzir
%um número de 4 dígitos, entre 1 e 6, que correspondem a uma jogada. Na janela de
%texto é dada a avaliação da jogada, indicando o número da jogada, o
%número introduzido, o número de dígitos certos na posição certa
%(indicados por um \textbf{X}) e de dígitos certos na posição errada
%(indicados por um \textbf{O}). Adicionalmente, na placa, os LED's indicam
%quanto tempo resta, o botão IA permite começar/reiniciar o jogo, o
%\textit{display} LCD indica a melhor pontuação até ao momento. O
%\textit{display} de 7 segmentos indica qual a jogada atual.

\section*{Visão Geral do Programa}
O programa está organizado nos seguintes módulos:
\begin{itemize}
\item \textbf{Módulo Check:} Avalia a jogada
  \begin{itemize}
  \item check\_play: Faz a verificação geral da jogada, colocando os
  argumentos na pilha e recolhendo o resultado para o por em memória.
  Permite que a função que gere o jogo seja mais legível.
  \item check: Faz a verificação da jogada, devolvendo o resultado.
  \end{itemize}

\item \textbf{Módulo Print:} Implementa funções para gerir a janela de
texto
  \begin{itemize}
  \item print: Imprime a jogada, colocando internamente os argumentos da
  função print\_line, permite que a função que gere o jogo seja mais
  legível.
  \item print\_str: Imprime uma string
  \item print\_line: Imprime a interação correspondente a uma jogada
  \item newline: Introduz uma nova linha
  \item limpa\_jt: Limpa a janela de texto
  \end{itemize}

\newpage
\item \textbf{Módulo Random:} Implementa funções para gerar números
pseudoaleatórios.
  \begin{itemize}
  \item random: Gera um número pseudoaletório, mediante uma semente
  disponível em memória
  \item get\_seed: Gera uma semente, correspondente ao valor do
  temporizador
  \end{itemize}
 
\item \textbf{Módulo Interface:} Gere a interação com o utilizador
  \begin{itemize}
  \item display: Atualiza todos os \textit{displays} da placa
  \item get\_input: Recolhe uma jogada do utilizador num tempo limite
  \end{itemize}
 
\item \textbf{Módulo Temporizador:} Implementa um temporizador
  \begin{itemize}
  \item start\_timer: Inicia o temporizador
  \item reset\_timer: Coloca o temporizador a zero e deixa-o desativado
  \end{itemize}
 
\item \textbf{Módulo Jogo:} Implementa a estrutura geral do jogo
  \begin{itemize}
  \item ronda: Permite jogar uma ronda
  \item update\_best: Atualiza o recorde
  \end{itemize}
 


\end{itemize}
\section*{Aspetos importantes da Implementação}
Como descrito no enunciado, as jogadas estão codificadas em 12 bits. O
resultado das jogadas encontra-se codificado em 16 bits, sendo o octeto
mais significativo o número de \textbf{O's} e o menos significativo o
número de \textbf{X's}. As funções preservam os valores dos registos que
usam, sendo, por isso, estanques.

Os valores relevantes para a gestão do jogo (como por exemplo, o
contador da jogada, o recorde, a jogada, o resultado) e para a gestão
adicionais(como por exemplo, o temporizador e a semente) estão
armazenados em memória.

\section*{Conclusão}
O programa foi implementado de forma a que o
nível de dificuldade (i.e. o tempo disponível para introduzir  uma
jogada e o número de jogadas de uma ronda) seja facilmente alteráveis
mudando o valor das constantes (NROUNDS\_LOST, NROUNDS\_MAX e
TEMP\_STEP). Para além disso não foram implementadas funções adicionais,
nem foram observadas divergências ao enunciado.
No entanto, há diversas funções (em particular os módulos Print e Random), que
podem ser reutilizadas.


\end{document}
